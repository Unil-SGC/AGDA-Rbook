% Options for packages loaded elsewhere
\PassOptionsToPackage{unicode}{hyperref}
\PassOptionsToPackage{hyphens}{url}
%
\documentclass[
]{book}
\usepackage{amsmath,amssymb}
\usepackage{iftex}
\ifPDFTeX
  \usepackage[T1]{fontenc}
  \usepackage[utf8]{inputenc}
  \usepackage{textcomp} % provide euro and other symbols
\else % if luatex or xetex
  \usepackage{unicode-math} % this also loads fontspec
  \defaultfontfeatures{Scale=MatchLowercase}
  \defaultfontfeatures[\rmfamily]{Ligatures=TeX,Scale=1}
\fi
\usepackage{lmodern}
\ifPDFTeX\else
  % xetex/luatex font selection
\fi
% Use upquote if available, for straight quotes in verbatim environments
\IfFileExists{upquote.sty}{\usepackage{upquote}}{}
\IfFileExists{microtype.sty}{% use microtype if available
  \usepackage[]{microtype}
  \UseMicrotypeSet[protrusion]{basicmath} % disable protrusion for tt fonts
}{}
\makeatletter
\@ifundefined{KOMAClassName}{% if non-KOMA class
  \IfFileExists{parskip.sty}{%
    \usepackage{parskip}
  }{% else
    \setlength{\parindent}{0pt}
    \setlength{\parskip}{6pt plus 2pt minus 1pt}}
}{% if KOMA class
  \KOMAoptions{parskip=half}}
\makeatother
\usepackage{xcolor}
\usepackage{color}
\usepackage{fancyvrb}
\newcommand{\VerbBar}{|}
\newcommand{\VERB}{\Verb[commandchars=\\\{\}]}
\DefineVerbatimEnvironment{Highlighting}{Verbatim}{commandchars=\\\{\}}
% Add ',fontsize=\small' for more characters per line
\usepackage{framed}
\definecolor{shadecolor}{RGB}{248,248,248}
\newenvironment{Shaded}{\begin{snugshade}}{\end{snugshade}}
\newcommand{\AlertTok}[1]{\textcolor[rgb]{0.94,0.16,0.16}{#1}}
\newcommand{\AnnotationTok}[1]{\textcolor[rgb]{0.56,0.35,0.01}{\textbf{\textit{#1}}}}
\newcommand{\AttributeTok}[1]{\textcolor[rgb]{0.13,0.29,0.53}{#1}}
\newcommand{\BaseNTok}[1]{\textcolor[rgb]{0.00,0.00,0.81}{#1}}
\newcommand{\BuiltInTok}[1]{#1}
\newcommand{\CharTok}[1]{\textcolor[rgb]{0.31,0.60,0.02}{#1}}
\newcommand{\CommentTok}[1]{\textcolor[rgb]{0.56,0.35,0.01}{\textit{#1}}}
\newcommand{\CommentVarTok}[1]{\textcolor[rgb]{0.56,0.35,0.01}{\textbf{\textit{#1}}}}
\newcommand{\ConstantTok}[1]{\textcolor[rgb]{0.56,0.35,0.01}{#1}}
\newcommand{\ControlFlowTok}[1]{\textcolor[rgb]{0.13,0.29,0.53}{\textbf{#1}}}
\newcommand{\DataTypeTok}[1]{\textcolor[rgb]{0.13,0.29,0.53}{#1}}
\newcommand{\DecValTok}[1]{\textcolor[rgb]{0.00,0.00,0.81}{#1}}
\newcommand{\DocumentationTok}[1]{\textcolor[rgb]{0.56,0.35,0.01}{\textbf{\textit{#1}}}}
\newcommand{\ErrorTok}[1]{\textcolor[rgb]{0.64,0.00,0.00}{\textbf{#1}}}
\newcommand{\ExtensionTok}[1]{#1}
\newcommand{\FloatTok}[1]{\textcolor[rgb]{0.00,0.00,0.81}{#1}}
\newcommand{\FunctionTok}[1]{\textcolor[rgb]{0.13,0.29,0.53}{\textbf{#1}}}
\newcommand{\ImportTok}[1]{#1}
\newcommand{\InformationTok}[1]{\textcolor[rgb]{0.56,0.35,0.01}{\textbf{\textit{#1}}}}
\newcommand{\KeywordTok}[1]{\textcolor[rgb]{0.13,0.29,0.53}{\textbf{#1}}}
\newcommand{\NormalTok}[1]{#1}
\newcommand{\OperatorTok}[1]{\textcolor[rgb]{0.81,0.36,0.00}{\textbf{#1}}}
\newcommand{\OtherTok}[1]{\textcolor[rgb]{0.56,0.35,0.01}{#1}}
\newcommand{\PreprocessorTok}[1]{\textcolor[rgb]{0.56,0.35,0.01}{\textit{#1}}}
\newcommand{\RegionMarkerTok}[1]{#1}
\newcommand{\SpecialCharTok}[1]{\textcolor[rgb]{0.81,0.36,0.00}{\textbf{#1}}}
\newcommand{\SpecialStringTok}[1]{\textcolor[rgb]{0.31,0.60,0.02}{#1}}
\newcommand{\StringTok}[1]{\textcolor[rgb]{0.31,0.60,0.02}{#1}}
\newcommand{\VariableTok}[1]{\textcolor[rgb]{0.00,0.00,0.00}{#1}}
\newcommand{\VerbatimStringTok}[1]{\textcolor[rgb]{0.31,0.60,0.02}{#1}}
\newcommand{\WarningTok}[1]{\textcolor[rgb]{0.56,0.35,0.01}{\textbf{\textit{#1}}}}
\usepackage{longtable,booktabs,array}
\usepackage{calc} % for calculating minipage widths
% Correct order of tables after \paragraph or \subparagraph
\usepackage{etoolbox}
\makeatletter
\patchcmd\longtable{\par}{\if@noskipsec\mbox{}\fi\par}{}{}
\makeatother
% Allow footnotes in longtable head/foot
\IfFileExists{footnotehyper.sty}{\usepackage{footnotehyper}}{\usepackage{footnote}}
\makesavenoteenv{longtable}
\usepackage{graphicx}
\makeatletter
\def\maxwidth{\ifdim\Gin@nat@width>\linewidth\linewidth\else\Gin@nat@width\fi}
\def\maxheight{\ifdim\Gin@nat@height>\textheight\textheight\else\Gin@nat@height\fi}
\makeatother
% Scale images if necessary, so that they will not overflow the page
% margins by default, and it is still possible to overwrite the defaults
% using explicit options in \includegraphics[width, height, ...]{}
\setkeys{Gin}{width=\maxwidth,height=\maxheight,keepaspectratio}
% Set default figure placement to htbp
\makeatletter
\def\fps@figure{htbp}
\makeatother
\setlength{\emergencystretch}{3em} % prevent overfull lines
\providecommand{\tightlist}{%
  \setlength{\itemsep}{0pt}\setlength{\parskip}{0pt}}
\setcounter{secnumdepth}{5}
\usepackage{booktabs}
\ifLuaTeX
  \usepackage{selnolig}  % disable illegal ligatures
\fi
\usepackage[]{natbib}
\bibliographystyle{apalike}
\IfFileExists{bookmark.sty}{\usepackage{bookmark}}{\usepackage{hyperref}}
\IfFileExists{xurl.sty}{\usepackage{xurl}}{} % add URL line breaks if available
\urlstyle{same}
\hypersetup{
  pdftitle={Advanced Geospatial Data Analysis in R: Environmental Application},
  pdfauthor={Marj Tonini, Haokun Liu},
  hidelinks,
  pdfcreator={LaTeX via pandoc}}

\title{Advanced Geospatial Data Analysis in R: Environmental Application}
\author{Marj Tonini, Haokun Liu}
\date{2023-11-21}

\usepackage{amsthm}
\newtheorem{theorem}{Theorem}[chapter]
\newtheorem{lemma}{Lemma}[chapter]
\newtheorem{corollary}{Corollary}[chapter]
\newtheorem{proposition}{Proposition}[chapter]
\newtheorem{conjecture}{Conjecture}[chapter]
\theoremstyle{definition}
\newtheorem{definition}{Definition}[chapter]
\theoremstyle{definition}
\newtheorem{example}{Example}[chapter]
\theoremstyle{definition}
\newtheorem{exercise}{Exercise}[chapter]
\theoremstyle{definition}
\newtheorem{hypothesis}{Hypothesis}[chapter]
\theoremstyle{remark}
\newtheorem*{remark}{Remark}
\newtheorem*{solution}{Solution}
\begin{document}
\maketitle

{
\setcounter{tocdepth}{1}
\tableofcontents
}
\hypertarget{about}{%
\chapter{About}\label{about}}

This is a \emph{sample} book written in \textbf{Markdown}. You can use anything that Pandoc's Markdown supports; for example, a math equation \(a^2 + b^2 = c^2\).

\hypertarget{usage}{%
\section{Usage}\label{usage}}

Each \textbf{bookdown} chapter is an .Rmd file, and each .Rmd file can contain one (and only one) chapter. A chapter \emph{must} start with a first-level heading: \texttt{\#\ A\ good\ chapter}, and can contain one (and only one) first-level heading.

Use second-level and higher headings within chapters like: \texttt{\#\#\ A\ short\ section} or \texttt{\#\#\#\ An\ even\ shorter\ section}.

The \texttt{index.Rmd} file is required, and is also your first book chapter. It will be the homepage when you render the book.

\hypertarget{render-book}{%
\section{Render book}\label{render-book}}

You can render the HTML version of this example book without changing anything:

\begin{enumerate}
\def\labelenumi{\arabic{enumi}.}
\item
  Find the \textbf{Build} pane in the RStudio IDE, and
\item
  Click on \textbf{Build Book}, then select your output format, or select ``All formats'' if you'd like to use multiple formats from the same book source files.
\end{enumerate}

Or build the book from the R console:

\begin{Shaded}
\begin{Highlighting}[]
\NormalTok{bookdown}\SpecialCharTok{::}\FunctionTok{render\_book}\NormalTok{()}
\end{Highlighting}
\end{Shaded}

To render this example to PDF as a \texttt{bookdown::pdf\_book}, you'll need to install XeLaTeX. You are recommended to install TinyTeX (which includes XeLaTeX): \url{https://yihui.org/tinytex/}.

\hypertarget{preview-book}{%
\section{Preview book}\label{preview-book}}

As you work, you may start a local server to live preview this HTML book. This preview will update as you edit the book when you save individual .Rmd files. You can start the server in a work session by using the RStudio add-in ``Preview book'', or from the R console:

\begin{Shaded}
\begin{Highlighting}[]
\NormalTok{bookdown}\SpecialCharTok{::}\FunctionTok{serve\_book}\NormalTok{()}
\end{Highlighting}
\end{Shaded}

\hypertarget{introduction-to-r}{%
\chapter{Introduction to R}\label{introduction-to-r}}

All chapters start with a first-level heading followed by your chapter title, like the line above. There should be only one first-level heading (\texttt{\#}) per .Rmd file.

\hypertarget{r-language}{%
\section{R Language}\label{r-language}}

R is a complete programming language and software environment for statistical computing and graphical representation.
As part of the GNU Project (free software, mass collaboration project), the source code is free available.
Its functionalists can be expanded by importing packages.
For more details on R see \url{https://www.r-project.org/}.

\hypertarget{r-packages}{%
\subsection{R Packages}\label{r-packages}}

A package is a file generally composed of R scripts (e.g., functions).
On all operation systems the function ``install.packages()'' can be used to download and install a package automatically.
Once a package has been installed, it can be loaded in a session by using the command \textcolor{red}{library(package)}.
To check the list of the installed libraries, the function \textcolor{red}{library()} can be used.
When you open an \textbf{R Markdown} document (.Rmd) the program propose you automatically to install the libraries listed there.

\hypertarget{some-tips}{%
\subsection{Some tips}\label{some-tips}}

\begin{itemize}
\tightlist
\item
  R is case sensitive!
\item
  Previously used command can be recalled in the console by using the \emph{up arrow} on the keyboard.
\item
  The working directory by default is ``\emph{C:/user/\ldots/Documents}''.

  \begin{itemize}
  \tightlist
  \item
    It can be found using the command \textcolor{red}{getwd()}
  \item
    It can be changed using the command line \textcolor{red}{setwd("C:/Your/own/path")}
  \end{itemize}
\item
  In \textbf{R Markdown}: the working directory when evaluating R code chunks is the directory of the input document by default.

  \begin{itemize}
  \tightlist
  \item
    To access to a specific file in a sub-folder use ``. /subfolder/file.ext''
  \item
    To access to a specific file in a up-folder use ``. . /upfolder/file.ext''
  \end{itemize}
\end{itemize}

\hypertarget{r-commands-online-resources}{%
\subsection{R Commands (online resources)}\label{r-commands-online-resources}}

Many table resuming the main R commands can be found online.
Here some useful links:

\begin{itemize}
\item
  \href{https://www.maths.usyd.edu.au/u/jchan/Rcommands.pdf}{A short list of the most useful R commands}
\item
  \href{https://sites.calvin.edu/scofield/courses/m143/materials/RcmdsFromClass.pdf}{Table of Useful R commands}
\item
  \href{https://rpubs.com/ssammut/ResearchStats}{Basic Commands to Get Started with R}
\end{itemize}

\hypertarget{r-markdown}{%
\section{R Markdown}\label{r-markdown}}

This is an R Markdown document :-)

Markdown is a simple formatting syntax for authoring HTML, PDF, and MS Word documents.
It is a simple and easy to use \textbf{plain text language} used to combine R code, results from your data analysis (including plots and tables), and written commentary into a single nicely formatted and reproducible document (like a report, publication, thesis chapter or a web pages).

Code lines are organized as code block, seeking to solve e specified task, and referred to as \textbf{``code chunk''}.
For more details on using R Markdown see \url{http://rmarkdown.rstudio.com}.

All what you have to do during the computing labs is to read each explanatory paragraph before running each individual R code chunk, one by one, and to interpret the results.
Finally, to create a personal document (usually PDF) from rmarkdown, you need to \textbf{Knit} the document.
Knitting a document simply means taking all the text and code and creating a nicely formatted document.

\hypertarget{data-type-in-computational-analysis}{%
\section{Data type in computational analysis}\label{data-type-in-computational-analysis}}

\hypertarget{variables}{%
\subsection{Variables}\label{variables}}

Variables are used to store values in a computer program.
Values can be numbers (real and complex), words (string), matrices, and even tables.

The fundamental or atomic data in R Programming can be:

\begin{itemize}
\tightlist
\item
  \textbf{integer}: number without decimals
\item
  \textbf{numeric}: number with decimals (float or double depending on the precision)
\item
  \textbf{character}: string, label
\item
  \textbf{factors}: a label with a limited number of categories
\item
  \textbf{logical}: true/false
\end{itemize}

\begin{figure}

{\centering \includegraphics[width=1\linewidth,height=0.8\textheight]{images/Rvariablesdata} 

}

\caption{Data Types in R \label{data_Type}}\label{fig:img1}
\end{figure}

\newpage

\hypertarget{data-structure-in-r}{%
\subsection{Data structure in R}\label{data-structure-in-r}}

R's base data structures can be organised by their dimensionality (1d, 2d, or nd) and whether they are homogeneous (all contents must be of the same type) or heterogeneous (the contents can be of different types).

This gives rise to the four data structures most often used in data analysis:

\begin{figure}

{\centering \includegraphics[width=0.5\linewidth,height=0.6\textheight]{images/data_type} 

}

\caption{Data structures in R \label{data_str}}\label{fig:img2}
\end{figure}

A \textbf{Vector} is a one-dimensional structure winch can contain object of one type only: numerical (integer and double), character, and logical.

\begin{Shaded}
\begin{Highlighting}[]
\CommentTok{\# Investigate vector\textquotesingle{}s types:}

\NormalTok{v1 }\OtherTok{\textless{}{-}} \FunctionTok{c}\NormalTok{(}\FloatTok{0.5}\NormalTok{, }\FloatTok{0.7}\NormalTok{); v1; }\FunctionTok{typeof}\NormalTok{(v1)}
\CommentTok{\#\textgreater{} [1] 0.5 0.7}
\CommentTok{\#\textgreater{} [1] "double"}

\NormalTok{v2 }\OtherTok{\textless{}{-}}\FunctionTok{c}\NormalTok{(}\DecValTok{1}\SpecialCharTok{:}\DecValTok{10}\NormalTok{); v2; }\FunctionTok{typeof}\NormalTok{(v2)}
\CommentTok{\#\textgreater{}  [1]  1  2  3  4  5  6  7  8  9 10}
\CommentTok{\#\textgreater{} [1] "integer"}

\NormalTok{v3 }\OtherTok{\textless{}{-}} \FunctionTok{c}\NormalTok{(}\ConstantTok{TRUE}\NormalTok{, }\ConstantTok{FALSE}\NormalTok{); v3; }\FunctionTok{typeof}\NormalTok{(v3)}
\CommentTok{\#\textgreater{} [1]  TRUE FALSE}
\CommentTok{\#\textgreater{} [1] "logical"}

\NormalTok{v4 }\OtherTok{\textless{}{-}} \FunctionTok{c}\NormalTok{(}\StringTok{"Swiss"}\NormalTok{, }\StringTok{"Itay"}\NormalTok{, }\StringTok{"France"}\NormalTok{, }\StringTok{"Germany"}\NormalTok{); v4; }\FunctionTok{typeof}\NormalTok{(v4)}
\CommentTok{\#\textgreater{} [1] "Swiss"   "Itay"    "France"  "Germany"}
\CommentTok{\#\textgreater{} [1] "character"}
\end{Highlighting}
\end{Shaded}

\begin{Shaded}
\begin{Highlighting}[]
\CommentTok{\#Create a sequence from 0 to 5 with a step of 0.5:}

\NormalTok{v5 }\OtherTok{\textless{}{-}} \FunctionTok{seq}\NormalTok{(}\DecValTok{1}\NormalTok{, }\DecValTok{5}\NormalTok{, }\AttributeTok{by=}\FloatTok{0.5}\NormalTok{); v5; }\FunctionTok{typeof}\NormalTok{(v5)}
\CommentTok{\#\textgreater{} [1] 1.0 1.5 2.0 2.5 3.0 3.5 4.0 4.5 5.0}
\CommentTok{\#\textgreater{} [1] "double"}

\FunctionTok{length}\NormalTok{(v5)}
\CommentTok{\#\textgreater{} [1] 9}

\FunctionTok{summary}\NormalTok{(v5)}
\CommentTok{\#\textgreater{}    Min. 1st Qu.  Median    Mean 3rd Qu.    Max. }
\CommentTok{\#\textgreater{}       1       2       3       3       4       5}
\end{Highlighting}
\end{Shaded}

\begin{Shaded}
\begin{Highlighting}[]
\CommentTok{\#Extract the third element of the vector}
\NormalTok{v5[}\DecValTok{3}\NormalTok{]}
\CommentTok{\#\textgreater{} [1] 2}

\CommentTok{\#Exclude the third element from the vector and save as new vector}
\NormalTok{v5[}\SpecialCharTok{{-}}\DecValTok{3}\NormalTok{]}
\CommentTok{\#\textgreater{} [1] 1.0 1.5 2.5 3.0 3.5 4.0 4.5 5.0}
\NormalTok{w5}\OtherTok{\textless{}{-}}\NormalTok{v5[}\SpecialCharTok{{-}}\DecValTok{3}\NormalTok{]; w5}
\CommentTok{\#\textgreater{} [1] 1.0 1.5 2.5 3.0 3.5 4.0 4.5 5.0}
\end{Highlighting}
\end{Shaded}

A \textbf{Matrix} is a two-dimensional structure winch can contain object of one type only.
The function \textcolor{red}{matrix()} can be used to construct matrices with specific dimensions.

\begin{Shaded}
\begin{Highlighting}[]

\CommentTok{\# Matrix of elements equal to "zero" and dimension 2x5 }
\NormalTok{m1}\OtherTok{\textless{}{-}}\FunctionTok{matrix}\NormalTok{(}\DecValTok{0}\NormalTok{,}\DecValTok{2}\NormalTok{,}\DecValTok{5}\NormalTok{); m1  }\CommentTok{\#(two rows by five columns)}
\CommentTok{\#\textgreater{}      [,1] [,2] [,3] [,4] [,5]}
\CommentTok{\#\textgreater{} [1,]    0    0    0    0    0}
\CommentTok{\#\textgreater{} [2,]    0    0    0    0    0}

\CommentTok{\# Matrix of integer elements (1 to 12, 3x4) }
\NormalTok{m2}\OtherTok{\textless{}{-}}\FunctionTok{matrix}\NormalTok{(}\DecValTok{1}\SpecialCharTok{:}\DecValTok{12}\NormalTok{, }\DecValTok{3}\NormalTok{,}\DecValTok{4}\NormalTok{); m2 }
\CommentTok{\#\textgreater{}      [,1] [,2] [,3] [,4]}
\CommentTok{\#\textgreater{} [1,]    1    4    7   10}
\CommentTok{\#\textgreater{} [2,]    2    5    8   11}
\CommentTok{\#\textgreater{} [3,]    3    6    9   12}

\CommentTok{\# Extract the second row}
\NormalTok{m2[}\DecValTok{2}\NormalTok{, ]}
\CommentTok{\#\textgreater{} [1]  2  5  8 11}
\CommentTok{\# Extract the third column}
\NormalTok{m2[,}\DecValTok{3}\NormalTok{]}
\CommentTok{\#\textgreater{} [1] 7 8 9}
\CommentTok{\# Extract the the second element of the third column}
\NormalTok{m2[}\DecValTok{2}\NormalTok{,}\DecValTok{3}\NormalTok{]}
\CommentTok{\#\textgreater{} [1] 8}
\end{Highlighting}
\end{Shaded}

\hypertarget{data-frame}{%
\subsection{Data Frame}\label{data-frame}}

A \textbf{data frame} allows to collect data of different type.
All elements must have the same length.

A \textbf{list} is a more flexible structure since it can contain variables of different types and lengths.
Nevertheless, the preferred structure for statistical analyses and computation is the data frame.

It is a good practice to explore the data frame before performing further computation on the data.
This can be simply accomplished by using the commands \textcolor{red}{str} to explore the structure of the data and \textcolor{red}{summary} to display the summary statistics and quickly summarize the data.
For numerical vectors the command \textcolor{red}{hist()} can be used to plot the basic histogram of the given values.

\begin{Shaded}
\begin{Highlighting}[]
\CommentTok{\# Create the vectors with the variables}

\NormalTok{cities }\OtherTok{\textless{}{-}} \FunctionTok{c}\NormalTok{(}\StringTok{"Berlin"}\NormalTok{, }\StringTok{"New York"}\NormalTok{, }\StringTok{"Paris"}\NormalTok{, }\StringTok{"Tokyo"}\NormalTok{)}
\NormalTok{area }\OtherTok{\textless{}{-}} \FunctionTok{c}\NormalTok{(}\DecValTok{892}\NormalTok{, }\DecValTok{1214}\NormalTok{, }\DecValTok{105}\NormalTok{, }\DecValTok{2188}\NormalTok{)}
\NormalTok{population }\OtherTok{\textless{}{-}} \FunctionTok{c}\NormalTok{(}\FloatTok{3.4}\NormalTok{, }\FloatTok{8.1}\NormalTok{, }\FloatTok{2.1}\NormalTok{, }\FloatTok{12.9}\NormalTok{)}
\NormalTok{continent }\OtherTok{\textless{}{-}} \FunctionTok{c}\NormalTok{(}\StringTok{"Europe"}\NormalTok{, }\StringTok{"Norh America"}\NormalTok{, }\StringTok{"Europe"}\NormalTok{, }\StringTok{"Asia"}\NormalTok{)}
\end{Highlighting}
\end{Shaded}

\begin{Shaded}
\begin{Highlighting}[]
\CommentTok{\# Concatenate the vectors into a new data frame}
\NormalTok{df1 }\OtherTok{\textless{}{-}} \FunctionTok{data.frame}\NormalTok{(cities, area, population, continent)}
\NormalTok{df1}
\CommentTok{\#\textgreater{}     cities area population    continent}
\CommentTok{\#\textgreater{} 1   Berlin  892        3.4       Europe}
\CommentTok{\#\textgreater{} 2 New York 1214        8.1 Norh America}
\CommentTok{\#\textgreater{} 3    Paris  105        2.1       Europe}
\CommentTok{\#\textgreater{} 4    Tokyo 2188       12.9         Asia}

\CommentTok{\#Add a column (e.g., language spoken) using the command "cbind"}
\NormalTok{df2 }\OtherTok{\textless{}{-}} \FunctionTok{cbind}\NormalTok{ (df1, }\StringTok{"Language"} \OtherTok{=} \FunctionTok{c}\NormalTok{ (}\StringTok{"German"}\NormalTok{, }\StringTok{"English"}\NormalTok{, }\StringTok{"Freanch"}\NormalTok{, }\StringTok{"Japanese"}\NormalTok{))}
\NormalTok{df2}
\CommentTok{\#\textgreater{}     cities area population    continent Language}
\CommentTok{\#\textgreater{} 1   Berlin  892        3.4       Europe   German}
\CommentTok{\#\textgreater{} 2 New York 1214        8.1 Norh America  English}
\CommentTok{\#\textgreater{} 3    Paris  105        2.1       Europe  Freanch}
\CommentTok{\#\textgreater{} 4    Tokyo 2188       12.9         Asia Japanese}
\end{Highlighting}
\end{Shaded}

\begin{Shaded}
\begin{Highlighting}[]
\CommentTok{\#Explore the data frame}
\FunctionTok{str}\NormalTok{(df2) }\CommentTok{\# see the structure}
\CommentTok{\#\textgreater{} \textquotesingle{}data.frame\textquotesingle{}:    4 obs. of  5 variables:}
\CommentTok{\#\textgreater{}  $ cities    : chr  "Berlin" "New York" "Paris" "Tokyo"}
\CommentTok{\#\textgreater{}  $ area      : num  892 1214 105 2188}
\CommentTok{\#\textgreater{}  $ population: num  3.4 8.1 2.1 12.9}
\CommentTok{\#\textgreater{}  $ continent : chr  "Europe" "Norh America" "Europe" "Asia"}
\CommentTok{\#\textgreater{}  $ Language  : chr  "German" "English" "Freanch" "Japanese"}
\FunctionTok{summary}\NormalTok{(df2) }\CommentTok{\# compute basic statistics}
\CommentTok{\#\textgreater{}     cities               area          population    }
\CommentTok{\#\textgreater{}  Length:4           Min.   : 105.0   Min.   : 2.100  }
\CommentTok{\#\textgreater{}  Class :character   1st Qu.: 695.2   1st Qu.: 3.075  }
\CommentTok{\#\textgreater{}  Mode  :character   Median :1053.0   Median : 5.750  }
\CommentTok{\#\textgreater{}                     Mean   :1099.8   Mean   : 6.625  }
\CommentTok{\#\textgreater{}                     3rd Qu.:1457.5   3rd Qu.: 9.300  }
\CommentTok{\#\textgreater{}                     Max.   :2188.0   Max.   :12.900  }
\CommentTok{\#\textgreater{}   continent           Language        }
\CommentTok{\#\textgreater{}  Length:4           Length:4          }
\CommentTok{\#\textgreater{}  Class :character   Class :character  }
\CommentTok{\#\textgreater{}  Mode  :character   Mode  :character  }
\CommentTok{\#\textgreater{}                                       }
\CommentTok{\#\textgreater{}                                       }
\CommentTok{\#\textgreater{} }

\CommentTok{\# Use the symbol "$" to address a particular column}
\NormalTok{pop}\OtherTok{\textless{}{-}}\NormalTok{(df2}\SpecialCharTok{$}\NormalTok{population)}
\NormalTok{pop}
\CommentTok{\#\textgreater{} [1]  3.4  8.1  2.1 12.9}
\FunctionTok{hist}\NormalTok{(pop) }\CommentTok{\# plot the histogram}
\end{Highlighting}
\end{Shaded}

\includegraphics{01-intro_files/figure-latex/data-frame3-1.pdf}

\hypertarget{cross}{%
\chapter{Cross-references}\label{cross}}

Cross-references make it easier for your readers to find and link to elements in your book.

\hypertarget{chapters-and-sub-chapters}{%
\section{Chapters and sub-chapters}\label{chapters-and-sub-chapters}}

There are two steps to cross-reference any heading:

\begin{enumerate}
\def\labelenumi{\arabic{enumi}.}
\tightlist
\item
  Label the heading: \texttt{\#\ Hello\ world\ \{\#nice-label\}}.

  \begin{itemize}
  \tightlist
  \item
    Leave the label off if you like the automated heading generated based on your heading title: for example, \texttt{\#\ Hello\ world} = \texttt{\#\ Hello\ world\ \{\#hello-world\}}.
  \item
    To label an un-numbered heading, use: \texttt{\#\ Hello\ world\ \{-\#nice-label\}} or \texttt{\{\#\ Hello\ world\ .unnumbered\}}.
  \end{itemize}
\item
  Next, reference the labeled heading anywhere in the text using \texttt{\textbackslash{}@ref(nice-label)}; for example, please see Chapter \ref{cross}.

  \begin{itemize}
  \tightlist
  \item
    If you prefer text as the link instead of a numbered reference use: \protect\hyperlink{cross}{any text you want can go here}.
  \end{itemize}
\end{enumerate}

\hypertarget{captioned-figures-and-tables}{%
\section{Captioned figures and tables}\label{captioned-figures-and-tables}}

Figures and tables \emph{with captions} can also be cross-referenced from elsewhere in your book using \texttt{\textbackslash{}@ref(fig:chunk-label)} and \texttt{\textbackslash{}@ref(tab:chunk-label)}, respectively.

See Figure \ref{fig:nice-fig}.

\begin{Shaded}
\begin{Highlighting}[]
\FunctionTok{par}\NormalTok{(}\AttributeTok{mar =} \FunctionTok{c}\NormalTok{(}\DecValTok{4}\NormalTok{, }\DecValTok{4}\NormalTok{, .}\DecValTok{1}\NormalTok{, .}\DecValTok{1}\NormalTok{))}
\FunctionTok{plot}\NormalTok{(pressure, }\AttributeTok{type =} \StringTok{\textquotesingle{}b\textquotesingle{}}\NormalTok{, }\AttributeTok{pch =} \DecValTok{19}\NormalTok{)}
\end{Highlighting}
\end{Shaded}

\begin{figure}

{\centering \includegraphics[width=0.8\linewidth]{02-cross-refs_files/figure-latex/nice-fig-1} 

}

\caption{Here is a nice figure!}\label{fig:nice-fig}
\end{figure}

Don't miss Table \ref{tab:nice-tab}.

\begin{Shaded}
\begin{Highlighting}[]
\NormalTok{knitr}\SpecialCharTok{::}\FunctionTok{kable}\NormalTok{(}
  \FunctionTok{head}\NormalTok{(pressure, }\DecValTok{10}\NormalTok{), }\AttributeTok{caption =} \StringTok{\textquotesingle{}Here is a nice table!\textquotesingle{}}\NormalTok{,}
  \AttributeTok{booktabs =} \ConstantTok{TRUE}
\NormalTok{)}
\end{Highlighting}
\end{Shaded}

\begin{table}

\caption{\label{tab:nice-tab}Here is a nice table!}
\centering
\begin{tabular}[t]{rr}
\toprule
temperature & pressure\\
\midrule
0 & 0.0002\\
20 & 0.0012\\
40 & 0.0060\\
60 & 0.0300\\
80 & 0.0900\\
\addlinespace
100 & 0.2700\\
120 & 0.7500\\
140 & 1.8500\\
160 & 4.2000\\
180 & 8.8000\\
\bottomrule
\end{tabular}
\end{table}

\hypertarget{parts}{%
\chapter{Parts}\label{parts}}

You can add parts to organize one or more book chapters together. Parts can be inserted at the top of an .Rmd file, before the first-level chapter heading in that same file.

Add a numbered part: \texttt{\#\ (PART)\ Act\ one\ \{-\}} (followed by \texttt{\#\ A\ chapter})

Add an unnumbered part: \texttt{\#\ (PART\textbackslash{}*)\ Act\ one\ \{-\}} (followed by \texttt{\#\ A\ chapter})

Add an appendix as a special kind of un-numbered part: \texttt{\#\ (APPENDIX)\ Other\ stuff\ \{-\}} (followed by \texttt{\#\ A\ chapter}). Chapters in an appendix are prepended with letters instead of numbers.

\hypertarget{footnotes-and-citations}{%
\chapter{Footnotes and citations}\label{footnotes-and-citations}}

\hypertarget{footnotes}{%
\section{Footnotes}\label{footnotes}}

Footnotes are put inside the square brackets after a caret \texttt{\^{}{[}{]}}. Like this one \footnote{This is a footnote.}.

\hypertarget{citations}{%
\section{Citations}\label{citations}}

Reference items in your bibliography file(s) using \texttt{@key}.

For example, we are using the \textbf{bookdown} package \citep{R-bookdown} (check out the last code chunk in index.Rmd to see how this citation key was added) in this sample book, which was built on top of R Markdown and \textbf{knitr} \citep{xie2015} (this citation was added manually in an external file book.bib).
Note that the \texttt{.bib} files need to be listed in the index.Rmd with the YAML \texttt{bibliography} key.

The \texttt{bs4\_book} theme makes footnotes appear inline when you click on them. In this example book, we added \texttt{csl:\ chicago-fullnote-bibliography.csl} to the \texttt{index.Rmd} YAML, and include the \texttt{.csl} file. To download a new style, we recommend: \url{https://www.zotero.org/styles/}

The RStudio Visual Markdown Editor can also make it easier to insert citations: \url{https://rstudio.github.io/visual-markdown-editing/\#/citations}

\hypertarget{blocks}{%
\chapter{Blocks}\label{blocks}}

\hypertarget{equations}{%
\section{Equations}\label{equations}}

Here is an equation.

\begin{equation} 
  f\left(k\right) = \binom{n}{k} p^k\left(1-p\right)^{n-k}
  \label{eq:binom}
\end{equation}

You may refer to using \texttt{\textbackslash{}@ref(eq:binom)}, like see Equation \eqref{eq:binom}.

\hypertarget{theorems-and-proofs}{%
\section{Theorems and proofs}\label{theorems-and-proofs}}

Labeled theorems can be referenced in text using \texttt{\textbackslash{}@ref(thm:tri)}, for example, check out this smart theorem \ref{thm:tri}.

\begin{theorem}
\protect\hypertarget{thm:tri}{}\label{thm:tri}For a right triangle, if \(c\) denotes the \emph{length} of the hypotenuse
and \(a\) and \(b\) denote the lengths of the \textbf{other} two sides, we have
\[a^2 + b^2 = c^2\]
\end{theorem}

Read more here \url{https://bookdown.org/yihui/bookdown/markdown-extensions-by-bookdown.html}.

\hypertarget{callout-blocks}{%
\section{Callout blocks}\label{callout-blocks}}

The \texttt{bs4\_book} theme also includes special callout blocks, like this \texttt{.rmdnote}.

You can use \textbf{markdown} inside a block.

\begin{Shaded}
\begin{Highlighting}[]
\FunctionTok{head}\NormalTok{(beaver1, }\AttributeTok{n =} \DecValTok{5}\NormalTok{)}
\CommentTok{\#\textgreater{}   day time  temp activ}
\CommentTok{\#\textgreater{} 1 346  840 36.33     0}
\CommentTok{\#\textgreater{} 2 346  850 36.34     0}
\CommentTok{\#\textgreater{} 3 346  900 36.35     0}
\CommentTok{\#\textgreater{} 4 346  910 36.42     0}
\CommentTok{\#\textgreater{} 5 346  920 36.55     0}
\end{Highlighting}
\end{Shaded}

It is up to the user to define the appearance of these blocks for LaTeX output.

You may also use: \texttt{.rmdcaution}, \texttt{.rmdimportant}, \texttt{.rmdtip}, or \texttt{.rmdwarning} as the block name.

The R Markdown Cookbook provides more help on how to use custom blocks to design your own callouts: \url{https://bookdown.org/yihui/rmarkdown-cookbook/custom-blocks.html}

\hypertarget{sharing-your-book}{%
\chapter{Sharing your book}\label{sharing-your-book}}

\hypertarget{publishing}{%
\section{Publishing}\label{publishing}}

HTML books can be published online, see: \url{https://bookdown.org/yihui/bookdown/publishing.html}

\hypertarget{pages}{%
\section{404 pages}\label{pages}}

By default, users will be directed to a 404 page if they try to access a webpage that cannot be found. If you'd like to customize your 404 page instead of using the default, you may add either a \texttt{\_404.Rmd} or \texttt{\_404.md} file to your project root and use code and/or Markdown syntax.

\hypertarget{metadata-for-sharing}{%
\section{Metadata for sharing}\label{metadata-for-sharing}}

Bookdown HTML books will provide HTML metadata for social sharing on platforms like Twitter, Facebook, and LinkedIn, using information you provide in the \texttt{index.Rmd} YAML. To setup, set the \texttt{url} for your book and the path to your \texttt{cover-image} file. Your book's \texttt{title} and \texttt{description} are also used.

This \texttt{bs4\_book} provides enhanced metadata for social sharing, so that each chapter shared will have a unique description, auto-generated based on the content.

Specify your book's source repository on GitHub as the \texttt{repo} in the \texttt{\_output.yml} file, which allows users to view each chapter's source file or suggest an edit. Read more about the features of this output format here:

\url{https://pkgs.rstudio.com/bookdown/reference/bs4_book.html}

Or use:

\begin{Shaded}
\begin{Highlighting}[]
\NormalTok{?bookdown}\SpecialCharTok{::}\NormalTok{bs4\_book}
\end{Highlighting}
\end{Shaded}


  \bibliography{book.bib,packages.bib}

\end{document}
